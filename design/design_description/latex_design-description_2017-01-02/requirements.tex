%%%%%%%%%%%%%%%%%%%%%%%%%%%%%%%%%%%%%%%%%%%%%%%%%%
%%%%%%%%%%

% requirements
% describe users (including their domains) here in more detail
% user needs
% tasks

\section{Requirements}

% figures

% tables

%%%%%%%%%%%%%%%%%%%%%%%%%%%%%%%%%%%%%%%%%%%%%%%%%%
%%%%%%%%%%
% table concept

% table
\begin{table}[htbp]
\small
\centering
\begin{tabular}{ |m{1.9cm}|m{11cm}|m{1cm}|m{1.3cm}|m{1cm}| }
% column header
\hline
\textbf{Category}
& \textbf{Concept}
& \textbf{Expert}
& \textbf{Inter\-mediate}
& \textbf{Novice}
\\
% encoding
\hline
\multirow{3}{1.5cm}{\textbf{Encoding}}
& \textbf{Genes} encode \textbf{transcripts} via \textbf{transcription}.
& \textbf{X}
& \textbf{X}
& \textbf{X}
\\
\cline{2-5}
& \textbf{Transcripts} encode \textbf{proteins} via \textbf{translation}.
& \textbf{X}
& \textbf{X}
& \textbf{X}
\\
\cline{2-5}
& \textbf{Proteins} mediate both \textbf{transcription} and \textbf{translation}.
& \textbf{X}
& \textbf{X}
&
\\
% expression
\hline
\multirow{2}{1.5cm}{\textbf{Expression}}
& \textbf{Proteins} regulate the extent of both \textbf{transcription} and \textbf{translation} via interaction with \textbf{genes}, \textbf{transcripts}, and other \textbf{proteins}.
& \textbf{X}
& \textbf{X}
&
\\
\cline{2-5}
& \textbf{Metabolites} regulate the extent of both \textbf{transcription} and \textbf{translation} via interaction with \textbf{proteins}.
& \textbf{X}
&
&
\\
% reaction
\hline
\multirow{8}{1.5cm}{\textbf{Reaction}}
& \textbf{Reactions} involve a chemical change between \textbf{metabolites}.
& \textbf{X}
& \textbf{X}
& \textbf{X}
\\
\cline{2-5}
& \textbf{Reactants} are \textbf{metabolites} that participate in the start of a \textbf{reaction}.
& \textbf{X}
& \textbf{X}
& \textbf{X}
\\
\cline{2-5}
& \textbf{Products} are \textbf{metabolites} that participate in the end of a \textbf{reaction}.
& \textbf{X}
& \textbf{X}
& \textbf{X}
\\
\cline{2-5}
& Some \textbf{metabolites} participate in more \textbf{reactions} than others.
& \textbf{X}
& \textbf{X}
&
\\
\cline{2-5}
& \textbf{Reactions} have different \textbf{rates}.
& \textbf{X}
&
&
\\
\cline{2-5}
& \textbf{Reactions} have different \textbf{directions} and likelihoods/probabilities of \textbf{reversibility}.
& \textbf{X}
&
&
\\
\cline{2-5}
& \textbf{Pathways} are sets of multiple \textbf{reactions} that carry out major processes.
& \textbf{X}
& \textbf{X}
&
\\
\cline{2-5}
& \textbf{Reactions} between \textbf{metabolites} make a cooperative and continuous network.
& \textbf{X}
&
&
\\
% catalysis
\hline
\multirow{1}{1.5cm}{\textbf{Catalysis}}
& \textbf{Proteins} catalyze \textbf{reactions} between \textbf{metabolites}.
& \textbf{X}
& \textbf{X}
&
\\
% transport
\hline
\multirow{2}{1.5cm}{\textbf{Transport}}
& \textbf{Membranes} partition the eukaryotic \textbf{cell} into separate \textbf{compartments}.
& \textbf{X}
& \textbf{X}
& \textbf{X}
\\
\cline{2-5}
& \textbf{Proteins} mediate \textbf{transport} of \textbf{metabolites} between \textbf{compartments}.
& \textbf{X}
& \textbf{X}
&
\\
% regulation
\hline
\multirow{4}{1.5cm}{\textbf{Regulation}}
& Multiple copies of the same \textbf{transcript}, \textbf{protein}, or \textbf{metabolite} are present in the cell in dynamic \textbf{pools} of variable amounts.
& \textbf{X}
& \textbf{X}
&
\\
\cline{2-5}
& \textbf{Metabolites} regulate the extent of \textbf{catalysis} and \textbf{transport} via interaction with \textbf{proteins}.
& \textbf{X}
&
&
\\
\cline{2-5}
& \textbf{Proteins} regulate the extent of \textbf{catalysis} and \textbf{transport} via interaction with other \textbf{proteins}.
& \textbf{X}
& \textbf{X}
&
\\
\cline{2-5}
& Changes to a single \textbf{metabolite} or \textbf{reaction} can influence others even great distances away in the network.
& \textbf{X}
&
&
\\
% technology
\hline
\multirow{3}{1.5cm}{\textbf{Technology}}
& \textbf{'Omics} technologies identify modifications to \textbf{genes}, \textbf{transcripts}, or \textbf{proteins}.
& \textbf{X}
& \textbf{X}
&
\\
\cline{2-5}
& \textbf{'Omics} technologies measure the abundances of \textbf{transcripts}, \textbf{proteins}, or \textbf{metabolites} in a sample and thereby suggest the pool amount in a cell.
& \textbf{X}
& \textbf{X}
&
\\
\cline{2-5}
& To compare relative abundances from multiple samples, it is necessary to consider any differences in \textbf{conditions} or \textbf{parameters}.
& \textbf{X}
&
&
\\
% science
\hline
\multirow{1}{1.5cm}{\textbf{Science}}
& \textbf{Experiments} require specific controls to study specific variables.
& \textbf{X}
& \textbf{X}
&
\\
\hline
\end{tabular}
\caption{Fundamental biological, technological, and scientific concepts relevant to the study of metabolism.}
\label{table:concept}
\end{table}



% users
% diverse backgrounds
% diverse levels of interest in metabolism
% common need to design experiments and interpret results
% diverse computational knowledge (keep it simple)

The target \textbf{users} of this project's tool have diverse expertise and interest in the study of metabolism.
They are scientists, engineers, or clinicians in the domains of \textbf{biology, biotechnology, pharmacology, or medicine}.
\textbf{Experts} are scientists or engineers who understand both general biology and specific metabolism thoroughly.
Their interest is either to discover new processes within this system or to modify it for some therapeutic or synthetic benefit.
They understand relevant methods and technologies and know how to use them properly in experimentation.
\textbf{Intermediates} are scientists or engineers who understand general biology but are less familiar with specific metabolism.
Their interest is to consider some aspects of metabolism that relate to another subject.
They also understand relevant methods and technologies and know how to use them properly in experimentation.
\textbf{Novices} are scientists, engineers, or clinicians who understand only basic, general biology.
Their interest is also to consider some aspects of metabolism that relate to another subject.
Their interest may even be to profile the metabolism of a patient in order to diagnose a disease or prescribe an appropriate therapy.
They do not understand relevant technologies or how to use them properly in experimentation.
\textbf{Table \ref{table:concept}} describes how these categories of users understand fundamental concepts in the study of metabolism.

% user needs

For all users, the study of metabolism has some general requirements.
Broadly, users' goals are \textbf{1) to design experiments reliably} and \textbf{2) to interpret experimental results accurately}.
Both of these goals require appropriate consideration for the context of the subject in study.
Since \textbf{metabolism is a continuous system that functions cooperatively}, the real context is the entire system, but its vastness and complexity are inaccessible.
To handle vastness, \textbf{users need information in detail for narrow selections without neglecting the broad context}.
To handle complexity, \textbf{users need to explore different types of information in different ways}.
In both cases, they need \textbf{flexibility to match their interest}.

%%%%%%%%%%
%%%%%%%%%%
%%%%%%%%%%

% selection requirements

\subsection{Selection Requirements}
Metabolism is a vast system.
The user needs to select portions of the metabolic system that are sufficiently simple to conceptualize in detail yet sufficiently inclusive to give accurate context.
These selections need to match the user's interest appropriately.

\begin{enumerate}

\item \textbf{Select portions of the metabolic system by how they relate to a set of entities of interest.}
\\
The user has interest in single or multiple entities (genes, transcripts, proteins, or metabolites) and needs information about these entities and also about other entities that relate to them.
This selection must be appropriate for the user's custom, case-specific interest.
It must be independent of typical pathways or other categories since interest often traverses or transcends these pathways.

\item \textbf{Select portions of the metabolic system that either have or do not have some specific property or attribute.}
\\
Entities have many \textbf{biological properties}.
They also can have \textbf{attributes from experimental measurements} such as fold change in abundance or p-value.
The user needs to identify entities that have or do not have some specific properties or attributes.

\item \textbf{Select portions of the metabolic system both by their relations and by their properties or attributes.}
\\
Combining selection criteria can be much more specific.
In particular, the user might have a list of entities with properties and attributes.
The user might only have interest in entities with certain properties and attributes from this list, for which they need information and context.

\item \textbf{Refine selection by inclusion or exclusion of individual entities.}
\\
The user needs to define very specific selections.
Direct selection of individual entities might be necessary if more general selections are too ambiguous.

\end{enumerate}

%%%%%%%%%%
%%%%%%%%%%
%%%%%%%%%%

% exploration tasks

\subsection{Exploration Requirements}
Metabolism is a complex system in that many types of entities and relations have many types of properties.
There is a lot of information of different types.
The user needs to explore selections of the metabolic network in different ways to access different types of information according to her/his interest.

\begin{enumerate}

\item \textbf{Display summary information for the entire metabolic system and for the selection.}
\\
The user needs a \textbf{concise orientation to the context of the system}.
The user also needs to compare the smaller selection to the larger system.

\item \textbf{Represent selections of the metabolic system clearly.}
\\
The user needs to select any individual entity in the representation to either access information in more detail or to modify this entity's representation.
The user needs to recognize both near and distant relations between entities clearly.
The user needs to trace paths of relations between entities easily for both near and distant relations.
The representation must be able to visualize different properties of the entities in the context of their relations.
The representation must be automatic and robust in order to accommodate custom selections.
The representation must also be dynamic to allow the user to interact with entities and to enable the user to change the representation to illustrate different properties or attributes.

\item \textbf{Associate entities clearly with their properties or attributes.}
\\
Properties of entities are important aspects of the metabolic system.
The user needs to recognize a variety of these properties easily.
Also, the relations and properties of the metabolic system give important context for attributes from experimental measurements.
The user needs to analyze these attributes within the context of the metabolic system.

\item \textbf{Display information in detail for entities in response to user interest and interaction.}
\\
Even if it is just for a single entity or a few entities from the selection, the user needs to access some information in detail.
This information can be of different types.
The user needs some way to specify and access this information of interest.

\end{enumerate}

%%%%%%%%%%
%%%%%%%%%%
%%%%%%%%%%

% Practical Requirements

\subsection{Practical Requirements}

\begin{enumerate}

\item \textbf{Accommodate custom models of metabolic systems.}
\\ Metabolic systems vary in different species and even in different tissues or conditions in the same species.
The user needs to use this tool to study her/his custom model of the metabolic system.

\item \textbf{Maximize accessibility of the tool.}
\\ The user will be unlikely or unable to use a tool that requires installation of custom programs or packages.
The user will also be unlikely or unable to use a tool that requires computational resources beyond those of a typical personal computer.

\item \textbf{Simplify the user interface.}
\\ The user is familiar with and able to use common programs with graphical user interfaces such as office programs and internet browsers.
The user has little knowledge or skill in computational programming.
The user will be unlikely or unable to use a tool that uses a script-like interface.
The tool needs to have a simple and attractive graphical user interface.
User interaction with this interface must be intuitive.
The user is unlikely to consult documentation for the tool.

\item \textbf{Minimize the need for long-term maintenance.}
\\ There is no guarantee that this tool will have support for long-term maintenance.
The tool needs to be sufficiently functional and robust without maintenance for a period of about 5 years.

\end{enumerate}

%%%%%%%%%%
%%%%%%%%%%
%%%%%%%%%%
