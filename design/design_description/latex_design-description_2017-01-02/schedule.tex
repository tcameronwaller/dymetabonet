%%%%%%%%%%%%%%%%%%%%%%%%%%%%%%%%%%%%%%%%%%%%%%%%%%
%%%%%%%%%%

% Schedule

\section{Implementation Schedule}

This section gives some detail on the steps that will be necessary to implement this project.
These steps are very much not comprehensive.
Many additional steps will become obvious during implementation.
It is difficult to anticipate the duration of each step, since many problems might arise.
Still, this project has a main goal and schedule.
This project will implement a functional prototype of the tool that is suitable for trials with users by June 2017 (6 months time from preparation of this document).

\begin{enumerate}

\item \textbf{Organize information from the metabolic model.}
\\ It will be necessary to convert the information of the metabolic model from SBML format to JSON format in order to access it conveniently with a JavaScript program.

\item \textbf{Develop methods to interpret the metabolic model as a network.}
\\ The metabolic model provides collections of metabolites and reactions with properties.
These are the nodes of the network.
The reactions also have properties that designate their reactant and product metabolites.
This information describes the links of the network.
It will be necessary to develop custom methods to interpret this information from the model as a network.
For example, methods need to navigate between metabolites and reactions.
It will also be necessary to develop methods to \textbf{derive additional properties} from the network.
Counts of nodes and links and degrees and betweenness centralities of metabolite nodes are examples.
It will also be necessary to derive a property to distinguish between transport reactions and chemical reactions.
Some tools may already be available to provide some of this functionality for interpreting and analyzing the network.
In order to be useful, tools will need to be packages or libraries for JavaScript.

\item \textbf{Develop methods to query the network.}
\\ It will be necessary to query the network by topology and properties.
It will be necessary to develop appropriate methods for these queries.
Some tools may already be available to provide some of this functionality.
In order to be useful, tools will need to be packages or libraries for JavaScript.

\item \textbf{Design and develop the interface for queries against the network.}
\\ The \textbf{Query Interface} will enable the user to construct and execute queries and to select between subnetworks that these queries return.
The program's module for the \textbf{Query Interface} will support this functionality.
It will assemble and execute queries against the network from the metabolic model.
These queries will return subsets of the metabolic model.
The interface will display a list of these to the user.
When the user selects an element from the list, the module will pass the subset of the metabolic model to the program's modules for the \textbf{Detail Interface}, the \textbf{Navigation Interface}, and the \textbf{Exploration Interface}.
These modules may modify copies of the subsets from the \textbf{Query Interface}.
They will not modify the original subsets of the metabolic model from the module for the \textbf{Query Interface}.
These original subsets will persist so that the user can revert.

\item \textbf{Design and develop the interface and visualizations for summary and detail information about the network and individual entities.}
\\ The \textbf{Detail Interface} will display to the user various properties and attributes about the entire model of the network, selections of subnetworks, or individual entities.
The program's module for the \textbf{Detail Interface} will support this functionality.
It will derive additional properties from the network or subnetwork (counts of nodes and links, degrees, betweenness centralities).
It will also construct the appropriate visual representations.

\item \textbf{Design and develop the interface for changing the visual representation of properties and attributes on the network.}
\\ The \textbf{Navigation Interface} will enable the user to control the visual representation of the subnetwork.
It will also enable the user to control the representation of properties and attributes on the visual representation of the subnetwork.
The program's module for the \textbf{Navigation Interface} will support this functionality.

\item \textbf{Develop and refine the visual representation of the network.}
\\ The \textbf{Exploration Interface} will represent the subnetwork along with any properties or attributes visually.
It is a high priority for this visual representation to accommodate custom subnetworks and to portray these clearly.
Larger subnetworks might especially be challenges to represent clearly.
Also, representation of properties, such as cellular compartment, by positional groups of nodes may require extensive development and optimization.
The program's module for the \textbf{Exploration Interface} will support this functionality.

\end{enumerate}