%%%%%%%%%%%%%%%%%%%%%%%%%%%%%%%%%%%%%%%%%%%%%%%%%%
%%%%%%%%%%
% introduction

\section{Introduction}

% describe the scope/goal of the project

%metabolism
% metabolism is fundamental part of biological system.
% metabolism is a network.
% metabolism functions a a cooperative and continuous whole.
% metabolic defects drive many human diseases.

\textbf{Metabolism} is an integral part of the biological systems of cells, tissues, and organisms.
\textbf{Catabolic processes} degrade food to liberate energy and materials.
\textbf{Anabolic processes} utilize these materials and harness this energy to synthesize all cellular components.
In this way, \textbf{metabolism sustains life and all of its diverse processes}.
It is little wonder that metabolic defects contribute to the pathology of many human diseases, including obesity, diabetes, cardiovascular disease, and cancer \supercite{van_der_klaauw_hunger_2015, haslam_obesity_2005}.

\textbf{Metabolism is a cooperative and continuous system}.
In the cell, tens of thousands of different types of \textbf{genes, transcripts, proteins, and metabolites} exist in various amounts in separate compartments.
These \textbf{compartments partition the cell into environments with specific purposes}.
Genes and transcripts encode proteins, and proteins have many functions.
\textbf{Proteins transport metabolites between compartments and catalyze chemical reactions between metabolites}, performing these tasks with exceptional precision.
They control and regulate the rates and compartments of transports or reactions as well as which metabolites participate.
Metabolites, in turn, are the fuel and building blocks of cells.
All of these diverse functions influence each other such that \textbf{local modifications can impart pervasive effects throughout the system}.

% study of metabolism
% users

The study of metabolism is changing.
Traditionally this study involved experts who conducted \textbf{reductionist experiments} to characterize the functions of individual genes, transcripts, proteins, and metabolites.
Over many years, many of these experiments have contributed sufficient knowledge to consider the \textbf{greater perspective of pathways, processes and entire systems} \supercite{wang_systems_2015}.
Modern technologies, especially the \textbf{omics technologies (genomics, transcriptomics, proteomics, and metabolomics)} measure these systems at nearly comprehensive scales and provide rich information \supercite{zhang_forward_2015}.
Also, there is an increase in \textbf{interest and accessibility} such that the study of metabolism is becoming more \textbf{interdisciplinary}.
Scientists and engineers from diverse backgrounds consider metabolic mechanisms and phenotypes, and metabolic profiling in the clinic is an important goal towards \textbf{precision medicine} \supercite{beebe_sharpening_2016, benson_clinical_2016, tebani_omics-based_2016, collins_new_2015, wang_systems_2015, zhang_forward_2015, topol_individualized_2014}.

% challenges

These changes present challenges and opportunities.
The modern study of metabolism needs an \textbf{holistic perspective}, such as that of the discipline of \textbf{systems biology}.
Only this perspective sufficiently considers the \textbf{real context} in order to \textbf{design experiments reliably and interpret experimental results accurately}.
However, the vastness and complexity of metabolism exceed even the ability of experts to understand in its entirety.
Novice and expert investigators alike need tools to study the metabolic system.
The goal of this project is to develop \textbf{computational methods and a tool to support the study of metabolism}.
