%%%%%%%%%%%%%%%%%%%%%%%%%%%%%%%%%%%%%%%%%%%%%%%%%%
%%%%%%%%%%

% Chapter Proposal

\section{Project Information}
\noindent
\textbf{Project Title:} Profondeur\newline
\textbf{Project Website:} https://tcameronwaller.github.io/profondeur/\newline
\textbf{Project Repository:} https://github.com/tcameronwaller/profondeur\newline
\textbf{Author Name:} Thomas Cameron Waller\newline
\textbf{Author EMail:} cameron.waller@biochem.utah.edu\newline
\textbf{Author Identifier:} 00409400

\section{Introduction}

Metabolism sustains all of life's diverse processes, from the release of energy from food to the synthesis of genomes.
It is an integral part of the biological system of the cell with extensive connections throughout.
Given its central role, it is no surprise that \textbf{defects in metabolism drive the pathology of many human diseases\supercite{van_der_klaauw_hunger_2015, haslam_obesity_2005}.}
These include obesity, diabetes, cardiovascular disease, and cancer.
More and more studies in biological research consider metabolic phenotypes and mechanisms, and metabolic profiling is an important component of precision medicine \supercite{beebe_sharpening_2016, benson_clinical_2016, collins_new_2015, zhang_forward_2015, topol_individualized_2014}.
For these reasons \textbf{it is necessary to study metabolism from a holistic perspective, in its true context within the entire biological system.}

\textbf{The vast complexity of metabolism exceeds the ability of human investigators to consider it without supporting methods.} 
As part of the biological system of the cell, metabolism forms an intricate network of relationships between genes, transcripts, proteins, and metabolites.
In the human genome, approximately 70,000 genes encode 70,000 transcripts and 30,000 proteins \supercite{zhang_forward_2015}.
About 5,000 of these proteins are enzymes or transporters \supercite{swainston_recon_2016, pornputtapong_human_2015, thiele_community-driven_2013} that either catalyze chemical reactions or transport molecules between cells and sub-cellular compartments.
Many more proteins regulate metabolic processes via the expression and activity of relevant genes, transcripts, and proteins.
Approximately 30,000 endogenous small molecular metabolites \supercite{wishart_hmdb_2013} complete the collection.
All of these genes, transcripts, proteins, and metabolites relate to each other in multiple ways, forming the metabolic network.
This network varies in different tissues throughout the human body.
This stunning complexity makes it prohibitive to track all processes without advanced methods to organize and access the information of the metabolic network.
We argue that \textbf{many studies certainly miss interesting biological trends because the context is too elaborate}.

The author of this project has an interest in the study of metabolism.
His primary research project as a graduate researcher is in the visual analysis of the metabolic network.
Indeed this course project is a small-scale start to a bigger research project in collaboration with Alexander Lex, Jared Paul Rutter, and others.

\section{Project Objectives}

\textbf{The biological study of metabolism has potential to benefit the domains of biotechnology, pharmacology, and medicine.}
Biology's goal is to discover and describe life's functions and processes.
As a basic science this domain is primarily informative, and it enables applications in many other domains.
Biotechnology alters the metabolic systems of non-human hosts to synthesize hormones, antibiotics, or other drugs.
Pharmacology identifies effective drug targets within the biological system, considers the broad effects of drugs on this system, and even exploits subtle side-effects of available drugs for use in new therapies.
Knowledge of processes in health enables medicine to recognize perturbations, diagnose diseases, and propose appropriate therapies.
\textbf{The goal of this project is to develop methods that support the study of metabolism by users in these domains (biology, biotechnology, pharmacology, and medicine).}

In their study of metabolism, a common task in these domains is to \textbf{access information about biological entities and the relations between them}.
There are many different types of these \textbf{entities (genes, transcripts, proteins, metabolites)} and \textbf{relations (code, expression, regulation, transport, catalysis, reaction)}.
They each have their own specific properties.
In complex organisms, genes belong to chromosomes and chromosomal loci.
In eukaryotes, proteins and metabolites localize to sub-cellular compartments called organelles.
Many proteins have non-protein chemical cofactors that are essential for their functions.
Reactions between metabolites belong to metabolic pathways of anabolism or catabolism and also have chemical properties such as rates and reduction or oxidation.
A user might need to know any of this information for an entity, a relation, or combinations thereof.
For example, a user might need to know which gene encodes a transcript and protein, which protein regulates the transcription of a gene, which metabolites react via a protein's enzymatic catalysis, or which catabolic reactions in the mitochondrial compartment involve proteins that require a specific cofactor.
There are also many other possibilities.
\textbf{Although much of this information is within current knowledge and is available in public databases, the scale of the biological system renders this access nontrivial.}

Another common task in these domains is the \textbf{analysis of experimental data that corresponds to entities in the biological system}. Modern -omics technologies describe the biological system at nearly-comprehensive scales.
Available information includes mutations and copy variations in genes (genomics), abundance and splice variations of transcripts (transcriptomics), abundance and post-translational modifications of proteins (proteomics), and abundance of metabolites (metabolomics) \supercite{zhang_forward_2015}.
It is very important to analyze and interpret these data in the real context of the biological system.
For example, biological trends may only be apparent with consideration for network topology.
\textbf{Correct analysis and interpretation of these experimental data rely on knowledge of the biological system.}
This analysis becomes a great challenge at nearly-comprehensive scales.

The goal of this project is to develop computational methods for visual exploration and analysis of the metabolic network.
It will address some of the specific needs of these domains.
Many of the domain tasks correspond to queries against a database of network information.
Analysis of experimental data in the context of network topology will require custom, probabilistic algorithms.
Visual design will help the human user to recognize trends and patterns easily.

\noindent
\textbf{Summary}
\newline
\noindent
\textbf{Goal:} Develop computational methods for visual exploration and analysis of the metabolic network.
\newline
\noindent
\textbf{Users:} Research scientists and engineers in biology, biotechnology, pharmacology, and medicine
\newline
\noindent
\textbf{Benefit:} Advance knowledge of human health and ability to diagnose and treat human disease.
\newline
\noindent
\textbf{Domain Tasks:}
\newline
\noindent
1) Access information about biological entities and the relations between them.
\newline
\noindent
2) Analyze experimental data in the context of the metabolic network.

\section{Data}

This project involves two categories of data.
The first category includes the current knowledge about entities and relations in the biological system and their properties.
This information is available in multiple public databases.
This information is enormous and complex, and it is unlikely that all of the information will be available from a single source.
To maintain a reasonable scope, this project will use a minor subset of this information.
The second category is experimental data from -omics technologies.
This sort of data is available in multiple public databases.
This sort of data often has technical complications and requires processing.
Again, to maintain a reasonable scope, this project probably will not use this category of data.

\section{Data Processing}

Some aspects of this project will involve substantial data processing.
Information on the metabolic network will come from multiple databases.
It will be necessary to organize this information in an appropriate format and assemble an accessible database.
Experimental data from -omics technologies is also complex and sometimes requires extensive data processing before it is informative or interpretable.

\section{Visualization Design}

The prospective users of this project need to access information about biological entities and the relations between them.
This domain task of accessing information involves multiple abstract tasks.

\textbf{It is necessary to access information that matches the user's interest}.
With several types of entities, several types of relations, and many instances of each, the information of this network is vast and complex.
The user needs to consider subsets of the metabolic system in appropriate detail (according to interest) without neglecting potentially-relevant context.
Therein is a challenge, simplifying the information without losing important aspects.
This process of selecting a relevant subset of the network will likely require multiple levels of user interaction.
\textbf{Queries against a comprehensive database will allow the user to define her or his own interest in a fairly broad or general scope.}
These queries will need to accommodate many variables in order to provide sufficient specificity.
\textbf{A visual, interactive interface for dynamic queries will help the user to specify subsets of interest.}
Search algorithms will then select relevant subsets of the data, ranking several possibilities according to query criteria.
Then the user will be able to select from a list of search results.
\textbf{A visual interface will help the user to distinguish between search results and select whichever is most appropriate.}
After selecting a result from a query search, the user will need to refine the subset further according to her or his specific interest.
The user will need to select which types of entities and relations to represent.
The user might even need to select specific entities and relations to represent.
\textbf{A visual interface for a selection menu and an interactive visualization of the network data will help the user to refine interest subsets.}

\textbf{It is necessary to communicate the relevant information to the user clearly and efficiently.}
The user may need to study small or large subsets of the metabolic network at various degrees of detail.
These subsets will include a lot of information.
From these subsets, users need to recognize connectivity and many properties of entities and their relations.
\textbf{A visual representation of the network, such as a node-link diagram, will communicate this information to the user.}
This diagram will require a dynamic layout.
Alternative visual representations of the network might also work.
Also, due to the complexity of the metabolic network and the limitations of available visual channels on a two-dimensional display, it will probably be necessary to represent information in multiple views.
The user will be able to switch or scroll between views.
Animations and other strategies will help to preserve the user's sense of context through these transitions.

\textbf{It is necessary to associate experimental data with entities and relations from the metabolic network and to identify trends in these data.}
Keys (such as names of genes, proteins, or metabolites) will associate experimental data to specific entities in the database.
Search algorithms will then handle experimental data similarly to other properties or attributes to recognize pathways with significant differences.

\section{Must-Have Features}

\begin{itemize}
\item interactive selection menu to include or remove specific types of entities and relations
\item visual representation for subset of the metabolic network
\item interaction with the network representation to include or remove specific elements
\item visual representation of multiple types of properties on the network representation
\end{itemize}

\section{Optional Features}

This course project only entails 6 weeks of time for 1 developer.
To maintain a reasonable scope, several features will be optional.
These features will be more important for a longer-term, more complete project.

\begin{itemize}
\item comprehensive database for current knowledge of the metabolic network
\item dynamic query interface
\item search algorithms to identify pathways between entities matching query criteria
\item analytical algorithms to identify network aspects with interesting trends in experimental data
\item interface for selection from list of search results
\item visual representation of experimental data on the network representation
\end{itemize}

\section{Project Schedule}

This course project entails 6 weeks of time for 1 developer.
\newline
\newline
\noindent
\textbf{Weekly Goals}
\begin{enumerate}
\item 24 October - 30 October
\begin{itemize}
\item visual design of interface, especially network representation
\end{itemize}
\item 31 October - 6 November
\begin{itemize}
\item visual design of interface, especially network representation
\item collection, processing, and organization of network data
\item implementation of visual interface
\end{itemize}
\item 7 November - 13 November
\begin{itemize}
\item implementation of visual interface
\item milestone review
\end{itemize}
\item 14 November - 20 November
\begin{itemize}
\item implementation of visual interface
\item implementation of interaction
\end{itemize}
\item 21 November - 27 November
\begin{itemize}
\item implementation of visual interface
\item implementation of interaction
\end{itemize}
\item 28 November - 2 December
\begin{itemize}
\item screen-cast tutorial video
\item website
\item final submission
\end{itemize}
\end{enumerate}